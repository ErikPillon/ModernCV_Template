\usepackage[english]{babel} 
\usepackage[T1]{fontenc} 
\usepackage[utf8]{inputenc} 
\usepackage{amsmath,amssymb,amsthm,amsfonts}
\usepackage{stackrel}
\usepackage{thmtools}
\usepackage{indentfirst}
\usepackage{graphicx} % per inserire le immagini/grafici
\usepackage{floatflt}
\usepackage{caption}
\captionsetup{tableposition=top,figureposition=bottom,font=small,format=hang,labelfont={sf,bf}}
\usepackage{sidecap}
\usepackage{time}
\usepackage{mathrsfs}

\usepackage{url}
\usepackage{mathtools}
\usepackage{enumerate} %per le liste in lettere romane
\usepackage{empheq} %per evidenziare le formule
\usepackage{color}
\usepackage[usenames,dvipsnames,svgnames,table]{xcolor} % Required for specifying colors by name
\definecolor{ocre}{RGB}{52,177,201} % Define the orange color used for highlighting throughout the book
\definecolor{personal}{HTML}{FACFF7}
\definecolor{personal2}{HTML}{FFEEFF}

\usepackage[big]{titlesec} % Allows customization of titles
%% Font Settings
%\usepackage{avant} % Use the Avantgarde font for headings
%\usepackage{times} % Use the Times font for headings
\usepackage{microtype} % Slightly tweak font spacing for aesthetics
\usepackage{palatino} % font figo 
\usepackage[normalem]{ulem}%per barrare le parole

%------------------ PACCHETTO GEOMETRY------------------------
\usepackage{geometry}
\geometry{a4paper,top=3cm,bottom=3cm,left=3cm,right=3cm,heightrounded,bindingoffset=5mm}
%---------------end PACCHETTO GEOMETRY------------------------

\usepackage{tikz}
\usepackage{centernot}
\usepackage{bbm}

% EXERCISES                                             
\usepackage[auto-label]{exsheets}
\SetupExSheets{
	% the following is for the link to the solutions and viceversa
	question/pre-body-hook = {%
		\hyperlink{sol:\CurrentQuestionID}{Go to Solution
			\GetQuestionProperty{counter}{\CurrentQuestionID}}\par
	},
	solution/pre-hook = {
		\hypertarget{sol:\CurrentQuestionID}{}%
	} ,
	solution/pre-body-hook = {%
		\hyperref[qu:\CurrentQuestionID]{Back to Question
			\GetQuestionProperty{counter}{\CurrentQuestionID}}\par
	}
}
% Adjust the heading so there's are gap between heading and
% the previous paragraph a littler larger than the default:
\DeclareInstance{exsheets-heading}{block}{default}
{
	join   = { title[r,B]number[l,B](1ex,0pt) } ,
	attach =
	{
		main[l,vc]title[l,vc](0pt,0pt) ;
		main[r,vc]points[l,vc](\marginparsep,0pt)
	},
	above  = \baselineskip-.5ex ,
	below  = .5ex
}	
% END EXERCISES                                               



%%%%%%%%%%%%%%%%%%%%%%%%%%%
%% DEFINIZIONI DI TEOREMI/DEFINIZIONI ECC
\usepackage{wasysym}

\newtheoremstyle{blacknumex}% Theorem style name
{5pt}% Space above
{5pt}% Space below
{\normalfont}% Body font
{} % Indent amount
{\small\bf\sffamily}% Theorem head font
{\;}% Punctuation after theorem head
{0.25em}% Space after theorem head
{\small\sffamily{\tiny\ensuremath{\blacksquare}}\nobreakspace\thmname{#1}\nobreakspace\thmnumber{\@ifnotempty{#1}{}\@upn{#2}}% Theorem text (e.g. Theorem 2.1)
	\thmnote{\nobreakspace\the\thm@notefont\sffamily\bfseries---\nobreakspace#3.}}% Optional theorem note




\theoremstyle{definition}
\newtheorem*{mydef}{Definition}
\newtheorem*{notation}{Notation}
\newtheorem{res}{Result}

\theoremstyle{plain}
\newtheorem{thm}{Theorem}[chapter]
\newtheorem*{thmnn}{Theorem}
\newtheorem{lmm}[thm]{Lemma}
\newtheorem{corollary}[thm]{Corollary}
\newtheorem{prop}[thm]{Proposition}

\theoremstyle{remark}
\newtheorem*{remarkT}{Remark}
\newtheorem*{recall}{Recall}
%\newtheorem*{exmpl}{Example}
\newtheorem*{exampleT}{Example}
\newtheorem*{claim}{Claim}

%\theoremstyle{blacknumex}
\newtheorem{exercise}{Exercise}[chapter]


\usepackage{mdframed}

% Exercise box	  
\newmdenv[skipabove=7pt,
skipbelow=7pt,
rightline=false,
leftline=true,
topline=false,
bottomline=false,
backgroundcolor=ocre!10,
linecolor=ocre,
innerleftmargin=5pt,
innerrightmargin=5pt,
innertopmargin=5pt,
innerbottommargin=5pt,
leftmargin=0cm,
rightmargin=0cm,
linewidth=4pt]{eBox}

%example box
\newmdenv[skipabove=7pt,
skipbelow=7pt,
rightline=false,
leftline=false,
topline=true,
bottomline=true,
backgroundcolor=Peach!10,
linecolor=Lavender,
innerleftmargin=5pt,
innerrightmargin=5pt,
innertopmargin=5pt,
innerbottommargin=5pt,
leftmargin=0cm,
rightmargin=0cm,
linewidth=2pt]{rBox}

% Corollary and Remark box
\newmdenv[skipabove=7pt,
skipbelow=7pt,
rightline=false,
leftline=true,
topline=false,
bottomline=false,
linecolor=gray,
backgroundcolor=black!5,
innerleftmargin=5pt,
innerrightmargin=5pt,
innertopmargin=5pt,
leftmargin=0cm,
rightmargin=0cm,
linewidth=4pt,
innerbottommargin=5pt]{cBox}


\newenvironment{remark} %a bit brutal but seems working
{\pushQED{\qed}\renewcommand{\qedsymbol}{$ \LHD $}\remarkT}
{\popQED\endremarkT}

\newenvironment{rmk}{\begin{cBox}\begin{remark}}{\end{remark}\end{cBox}}


\newenvironment{rec} %a bit brutal but seems working
{\pushQED{\qed}\renewcommand{\qedsymbol}{$ \LHD $}\recall}
{\popQED\endrecall}


\newenvironment{cor}{\begin{cBox}\begin{corollary}}{\end{corollary}\end{cBox}}	

\newenvironment{result}{\begin{rBox}\begin{res}}{\end{res}\end{rBox}}
	
\newenvironment{ex}{\begin{eBox}\begin{exercise}}{\hfill{\color{ocre}\tiny\ensuremath{\blacksquare}}\end{exercise}\end{eBox}}				  

\newenvironment{exmpl}{\begin{rBox}\begin{exampleT}}{\hfill{\color{Lavender}\tiny\ensuremath{\blacksquare}}\end{exampleT}\end{rBox}}

%% RIDEFINISCO QUI DI SEGUITO ALCUNI COMANDI UTILI IN SEGUITO
\newcommand{\B}{\mathcal{B}}
\newcommand{\C}{\mathbb{C}}
\newcommand{\Hi}{\mathbb{H}}
\newcommand{\K}{\mathbb{K}}
\newcommand{\N}{\mathbb{N}}
\newcommand{\Q}{\mathbb{Q}}
\newcommand{\R}{\mathbb{R}}
\newcommand{\Rn}{\mathbb{R}^n}
\newcommand{\Rm}{\mathbb{R}^m}
\newcommand{\Rd}{\mathbb{R}^d}
\newcommand{\Z}{\mathbb{Z}}

\renewcommand{\leq}{\leqslant}
\renewcommand{\geq}{\geqslant}
\renewcommand{\epsilon}{\varepsilon}
\newcommand{\supp}{\operatorname{supp}}
\newcommand{\pd}[2]{\dfrac{\partial #1}{\partial #2}}
\newcommand{\diff}{\mathop{}\!d}
\newcommand{\Div}{\mbox{\rm div\,}}  %divergenza
\newcommand{\Rot}{\mbox{\rm rot\,}}  %rotore
\newcommand{\Grad}{\mbox{\rm grad\,}} % gradiente
\newcommand{\Supp}{\mbox{\rm supp\,}}  %supporto
\newcommand{\dist}{\mbox{\rm dist\,}}  %distance
\DeclareMathOperator{\spn}{span} %span
%\renewcommand{\phi}{\varphi}
\def\czero{{\cal C}^0}
\def\cuno{{\cal C}^1}
\def\cdue{{\cal C}^2}
\def\cinfinito{{\cal C}^\infty}
% STRAODIO L'UGUALE CON SOPRA IL PUNTO!
\renewcommand{\doteq}{:=}

% codici per modificare la norma e il valore assoluto
\DeclarePairedDelimiter{\abs}{\lvert}{\rvert}
\DeclarePairedDelimiter{\norma}{\lVert}{\rVert}
\newcommand{\prin}[2]{\langle #1,#2 \rangle} % per il prodotto interno
\newcommand{\paa}[2]{B(#1,#2[} %palla aperta
\newcommand{\pac}[2]{B(#1,#2]} %palla chiusa 
\newcommand{\intern}[1]{\text{int} #1 }
\newcommand{\La}[2]{\mathcal{L}(#1,#2)}
\makeatletter
\def\cleardoublepage{\clearpage\if@twoside\ifodd\c@page
	\else\hbox{}\thispagestyle{empty}\newpage\if@twocolumn\hbox{}\newpage\fi\fi\fi}
\makeatother
%%%%%%%%%%%%%%%%%%%%%%%%%%%%%%%%%%%%%%%%%%%

\usepackage{fancyhdr} % Required for header and footer configuration

\pagestyle{fancy}
\renewcommand{\chaptermark}[1]{\markboth{\sffamily\normalsize\bfseries\chaptername\ \thechapter.\ #1}{}} % Chapter text font settings
\renewcommand{\sectionmark}[1]{\markright{\sffamily\normalsize\bfseries\thesection\hspace{5pt}#1}{}} % Section text font settings
\fancyhf{} \fancyhead[LE,RO]{\sffamily\normalsize\thepage} % Font setting for the page number in the header
\fancyhead[LO]{\rightmark} % Print the nearest section name on the left side of odd pages
\fancyhead[RE]{\leftmark} % Print the current chapter name on the right side of even pages
\renewcommand{\headrulewidth}{0.5pt} % Width of the rule under the header
\addtolength{\headheight}{2.5pt} % Increase the spacing around the header slightly
\renewcommand{\footrulewidth}{0pt} % Removes the rule in the footer
\fancypagestyle{plain}{\fancyhead{}\renewcommand{\headrulewidth}{0pt}} % Style for when a plain pagestyle is specified


\usepackage{kantlipsum}

\usepackage{hyperref}
\hypersetup{colorlinks=true,
	linkcolor=blue,
	urlcolor=blue,
	citecolor=green,
	anchorcolor = yellow
}

%-------------------------------------------------------------------------------------
%	SECTION NUMBERING IN THE MARGIN
%-------------------------------------------------------------------------------------

\makeatletter
%\@startsection{<name>}{<level>}{<indent>}{<beforeskip>}{<afterskip>}{<style>}*[<altheading>]{<heading>}

%\renewcommand{\chapter}{\@startsection{chapter}{0}{1ex}
%	{-4ex \@plus -1ex \@minus -.4ex}
%	{1ex \@plus.2ex }
%	{\normalfont\Huge\sffamily\bfseries}}

\renewcommand{\@seccntformat}[1]{\llap{\textcolor{ocre}{\csname the#1\endcsname}\hspace{1em}}}                    
\renewcommand{\section}{\@startsection{section}{1}{\z@}
	{-4ex \@plus -1ex \@minus -.4ex}
	{1ex \@plus.2ex }
	{\normalfont\LARGE\sffamily\bfseries}}
\renewcommand{\subsection}{\@startsection {subsection}{2}{\z@}
	{-3ex \@plus -0.1ex \@minus -.4ex}
	{0.5ex \@plus.2ex }
	{\normalfont\Large\sffamily\bfseries}}
\renewcommand{\subsubsection}{\@startsection {subsubsection}{3}{\z@}
	{-2ex \@plus -0.1ex \@minus -.2ex}
	{.2ex \@plus.2ex }
	{\normalfont\small\sffamily\bfseries}}                        
\renewcommand\paragraph{\@startsection{paragraph}{4}{\z@}
	{-2ex \@plus-.2ex \@minus .2ex}
	{.1ex}
	{\normalfont\small\sffamily\bfseries}}
%\titleformat{\section}[block]{\LARGE\bfseries}{\thesection}{10pt}{\LARGE\bfseries}
\makeatother

%%%%%%%%%%%%%%%%%%%%%%%%%%%%%%%%%%%%%%%

\makeatletter
\def\@makechapterhead#1{%
	\vspace*{10\p@}% default: 50pt
	{\parindent \z@ \raggedright \normalfont
		\ifnum \c@secnumdepth >\m@ne
		%		\huge\bfseries \@chapapp\space \thechapter
		\fontsize{50}{60}\selectfont\bfseries\textcolor{ocre}{\thechapter}\space| \space 
		%\par\nobreak
		%\vskip 20\p@
		\fi
		\interlinepenalty\@M
		\Huge\sffamily\bfseries  #1\par\nobreak
		\vskip 40\p@
	}
}
\def\@makeschapterhead#1{%
	\vspace*{10\p@}% default: 50pt
	{\parindent \z@ \raggedright
		\normalfont
		\interlinepenalty\@M
		\sffamily\Huge\bfseries  #1\par\nobreak
		\vskip 40\p@
	}
}
\makeatother



\frenchspacing

\usepackage[some]{background}
\definecolor{titlepagecolor}{cmyk}{1,.60,0,.40}

\backgroundsetup{
	scale=1,
	angle=0,
	opacity=1,
	contents={\begin{tikzpicture}[remember picture,overlay]
		\path [fill=titlepagecolor] (current page.west)rectangle (current page.north east); 
		\draw [color=white, very thick] (5,0)--(5,0.5\paperheight);
		\end{tikzpicture}}
}
